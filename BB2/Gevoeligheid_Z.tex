\section{Bepaling Gevoeligheid $Z$}
Als eerste opdracht bepalen we de gevoeligheid $Z$ van de brug in functie van alle 9 mogelijke combinaties
van de weerstanden $R_A$ en $zR_B$. We veronderstellen hiervoor deze waarden:
\begin{itemize}
    \item $U = 1 V$
    \item $R_X = 120 \Omega$
    \item $R_g = 100 \Omega$ voor de 2 eerste gevallen ($1000 \Omega$ voor de andere 2)
    \item $\Delta I = 10^{-7}$
\end{itemize}
Voor de Wheatstone brug type A gebruiken we de volgende formule:
\begin{equation}
    Z = U \cdot \frac{R_A}{R_A + R_B} \cdot \frac{1}{R_x + R_A + R_g + \frac{R_A}{R_B}\cdot R_g}
\end{equation}
Voor type A verwisselen we $R$ en $R_A$ van plaats in de formule:
\begin{equation}
    Z = U \cdot \frac{R}{R + R_B} \cdot \frac{1}{R_x + R + R_g + \frac{R}{R_B}\cdot R_g}
\end{equation}
Verder berekenen we tevens:
\begin{itemize}
    \item Evenwichtsvoorwaarde voor de instelbare weerstand $R$
    \item Kleinste verandering $\Delta R$ van $R$ om de brug in evenwicht te brengen
    \item Overeenstemmende kleinst mogelijk detecteerbare variatie $\Delta R_x$ van de te meten $R_x$ 
    \item Stroom door $R_A$ en $R_B$ bij evenwicht van de brug
\end{itemize}
We krijgen dus voor de 4 tabellen:
\begin{table}
    \centering
    \label{tab:Lamflow}
    \caption{Gemeten waarden in het turbulent regime}
    \begin{tabular}{| c | c | c | c | c | c | c | c | c |}
        \hline
                & $R_A [\Omega]$    & $R_B [\Omega]$    & $R [\Omega]$  & $Z [A]$               & $\Delta R [\Omega]$   & $\Delta R_x [\Omega]$ & $I_x [A]$                 & $I_B [A]$             \\ \hline
        1       & 10                & 10                & 120           & $1.52 \cdot 10^{-3}$  & $7.92 \cdot 10^{-3} $ & $7.92 \cdot 10^{-3} $ & $4.917 \cdot 10^{-3}$     & $5.00 \cdot 10^{-2} $ \\ \hline
        2       & 10                & 100               & 1200          & $3.79 \cdot 10^{-4}$  & $3.17 \cdot 10^{-2}$  & $3.17 \cdot 10^{-2}$  & $7.58 \cdot 10^{-4}$      & $9.09 \cdot 10^{-3}$  \\ \hline   
        3       & 10                & 1000              & 12000         & $4.29 \cdot 10^{-4}$  & $2.80 \cdot 10^{-4}$  & $2.80 \cdot 10^{-4}$  & $8.25 \cdot 10^{-4}$      & $9.90 \cdot 10^{-4}$  \\ \hline
        4       & 100               & 10                & 12            & $6.89 \cdot 10^{-4}$  & $3.79 \cdot 10^{-4}$  & $3.79 \cdot 10^{-4}$  & $3.79 \cdot 10^{-4}$      & $3.79 \cdot 10^{-4}$  \\ \hline
    \end{tabular}
\end{table}