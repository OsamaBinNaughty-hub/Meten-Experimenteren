\section{Bepaling Gevoeligheid $Z$}
Als eerste opdracht bepalen we de gevoeligheid $Z$ van de brug in functie van alle 9 mogelijke combinaties
van de weerstanden $R_A$ en $zR_B$. We veronderstellen hiervoor deze waarden:
\begin{itemize}
    \item $U = 1 V$
    \item $R_X = 120 \Omega$
    \item $R_g = 100 \Omega$ voor de 2 eerste gevallen ($1000 \Omega$ voor de andere 2)
    \item $\Delta I = 10^{-7}$
\end{itemize}
Voor de Wheatstone brug type A gebruiken we de volgende formule:
\begin{equation}
    Z = U \cdot \frac{R_A}{R_A + R_B} \cdot \frac{1}{R_x + R_A + R_g + \frac{R_A}{R_B}\cdot R_g}
\end{equation}
Voor type A verwisselen we $R$ en $R_A$ van plaats in de formule:
\begin{equation}
    Z = U \cdot \frac{R}{R + R_B} \cdot \frac{1}{R_x + R + R_g + \frac{R}{R_B}\cdot R_g}
\end{equation}
Verder berekenen we tevens:
\begin{itemize}
    \item Evenwichtsvoorwaarde voor de instelbare weerstand $R$
    \item Kleinste verandering $\Delta R$ van $R$ om de brug in evenwicht te brengen
    \item Overeenstemmende kleinst mogelijk detecteerbare variatie $\Delta R_x$ van de te meten $R_x$ 
    \item Stroom door $R_A$ en $R_B$ bij evenwicht van de brug
\end{itemize}
We krijgen dus voor de 4 tabellen:
\begin{table}[H]
    \centering
    \label{tab:TA100}
    \caption{Theoretische waarden voor Type A bij 100 $\Omega$}
    \begin{tabular}{| c | c | c | c | c | c | c | c | c |}
        \hline
                & $R_A [\Omega]$    & $R_B [\Omega]$    & $R [\Omega]$  & $Z [A]$   & $\Delta R [\Omega]$   & $\Delta R_x [\Omega]$ & $I_x [A]$                 & $I_B [A]$             \\ \hline
        1       & 10                & 10                & 120           & 1,52E-03  & 7,92E-03              & 7,92E-03              & 4,17E-03                  & 5,00E-02 \\ \hline
        2       & 10                & 100               & 1200          & 3,79E-04  & 3,17E-02              & 3,17E-02              & 7,58E-04                  & 9,09E-03  \\ \hline   
        3       & 10                & 1000              & 12000         & 4,29E-05  & 2,80E+01              & 2,80E-01              & 8,25E-05                  & 9,90E-04  \\ \hline
        4       & 100               & 10                & 12            & 6,89E-04  & 1,74E-03              & 1,74E-02              & 7,58E-03                  & 9,09E-03  \\ \hline
        5	& 100	   &100	&120	&1,19E-03	&1,01E-02	&1,01E-02	&4,17E-03	&5,00E-03 \\ \hline
        6	&100	&1000	&1200	&2,75E-04	&4,36E-01	&4,36E-02	&7,58E-04	&9,09E-04 \\ \hline
        7	&1000	&10	&1,2	&8,82E-05	&1,36E-03	&1,36E-01	&8,25E-03	&9,90E-04 \\ \hline 
        8	&1000	&100	&12	&4,10E-04	&2,93E-03	&2,93E-02	&7,58E-03	&9,09E-04 \\ \hline
        9	&1000	&1000	&120	&3,79E-04	&3,17E-02	&3,17E-02	&4,17E-03	&5,00E-04 \\ \hline
    \end{tabular}
\end{table}
\begin{table}[H]
    \centering
    \label{tab:TA1OOO}
    \caption{Theoretische waarden voor Type A bij 1000 $\Omega$}
    \begin{tabular}{| c | c | c | c | c | c | c | c | c |}
        \hline
                & $R_A [\Omega]$    & $R_B [\Omega]$    & $R [\Omega]$  & $Z [A]$   & $\Delta R [\Omega]$   & $\Delta R_x [\Omega]$ & $I_x [A]$                 & $I_B [A]$             \\ \hline
                1	&10	&10	&120	&2,35E-04	&5,11E-02	&5,11E-02	&4,17E-03	&5,00E-02           \\ \hline
                2	&10	&100	&1200	&7,39E-05	&1,62E+00	&1,62E-01	&7,58E-04	&9,09E-03\\ \hline
                3	&10	&1000	&12000	&8,69E-06	&1,38E+02	&1,38E+00	&8,25E-05	&9,90E-04\\ \hline
                4	&100	&10	&12	&8,10E-05	&1,48E-02	&1,48E-01	&7,58E-03	&9,09E-03\\ \hline
                5	&100	&100	&120	&2,25E-04	&5,33E-02	&5,33E-02	&4,17E-03	&5,00E-03\\ \hline
                6	&100	&1000	&1200	&6,89E-05	&1,74E+00	&1,74E-01	&7,58E-04	&9,09E-04\\ \hline
                7	&1000	&10	&1,2	&9,70E-06	&1,24E-02	&1,24E+00	&8,25E-03	&9,90E-04\\ \hline
                8	&1000	&100	&12	&7,50E-05	&1,60E-02	&1,60E-01	&7,58E-03	&9,09E-04\\ \hline
                9	&1000	&1000	&120	&1,60E-04	&7,49E-02	&7,49E-02	&4,17E-03	&5,00E-04\\ \hline
    \end{tabular}
\end{table}
\begin{table}[H]
    \centering
    \label{tab:TB1OO}
    \caption{Theoretische waarden voor Type B bij 100 $\Omega$}
    \begin{tabular}{| c | c | c | c | c | c | c | c | c |}
        \hline
                & $R_A [\Omega]$    & $R_B [\Omega]$    & $R [\Omega]$  & $Z [A]$   & $\Delta R [\Omega]$   & $\Delta R_x [\Omega]$ & $I_x [A]$                 & $I_B [A]$             \\ \hline
                1	&10	&10	&120	&5,99E-04	&1,67E-03	&2,00E-02	&7,69E-03	&7,69E-03 \\ \hline
                2	&10	&100	&1200	&3,52E-04	&2,84E-03	&3,41E-02	&7,69E-03	&7,69E-04\\ \hline
                3	&10	&1000	&12000	&6,88E-05	&1,45E-02	&1,74E-01	&8,25E-04	&7,69E-05\\ \hline
                4	&100	&10	&12	&1,55E-03	&6,45E-03	&7,74E-03	&4,55E-03	&4,55E-02\\ \hline
                5	&100	&100	&120	&1,19E-03	&8,43E-03	&1,01E-02	&4,55E-03	&4,55E-03\\ \hline
                6	&100	&1000	&1200	&3,54E-04	&2,82E-02	&3,39E-02	&4,55E-03	&4,55E-04\\ \hline
                7	&1000	&10	&1,2	&4,59E-04	&2,18E-01	&2,61E-02	&8,93E-04	&8,93E-02\\ \hline
                8	&1000	&100	&12	&4,39E-04	&2,28E-01	&2,73E-02	&8,93E-04	&8,93E-03\\ \hline
                9	&1000	&1000	&120	&3,04E-04	&3,29E-01	&3,94E-02	&8,93E-04	&8,93E-04     \\ \hline  
    \end{tabular}
\end{table}
\begin{table}[H]
    \centering
    \label{tab:TB1OO0}
    \caption{Theoretische waarden voor Type B bij 1000 $\Omega$}
    \begin{tabular}{| c | c | c | c | c | c | c | c | c |}
        \hline
                & $R_A [\Omega]$    & $R_B [\Omega]$    & $R [\Omega]$  & $Z [A]$   & $\Delta R [\Omega]$   & $\Delta R_x [\Omega]$ & $I_x [A]$                 & $I_B [A]$             \\ \hline
                1	&10	&10	&120	&6,97E-05	&1,43E-02	&1,72E-01	&4,17E-03	&7,69E-03 \\ \hline
                2	&10	&100	&1200	&6,45E-05	&1,55E-02	&1,86E-01	&7,58E-04	&7,69E-04\\ \hline
                3	&10	&1000	&12000	&3,67E-05	&2,72E-02	&3,27E-01	&8,25E-05	&7,69E-05\\ \hline
                4	&100	&10	&12	&2,34E-04	&4,28E-02	&5,13E-02	&7,58E-03	&4,55E-02\\ \hline
                5	&100	&100	&120	&2,24E-04	&4,47E-02	&5,37E-02	&4,17E-03	&4,55E-03\\ \hline
                6	&100	&1000	&1200	&1,55E-04	&6,45E-02	&7,74E-02	&7,58E-04	&4,55E-04\\ \hline
                7	&1000	&10	&1,2	&8,63E-05	&1,16E+00	&1,39E-01	&8,25E-03	&8,93E-02\\ \hline
                8	&1000	&100	&12	&8,56E-05	&1,17E+00	&1,40E-01	&7,58E-03	&8,93E-03\\ \hline
                9	&1000	&1000	&120	&7,88E-05	&1,27E+00	&1,52E-01	&4,17E-03	&8,93E-04  \\ \hline
    \end{tabular}
\end{table}