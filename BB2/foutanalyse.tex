\section{Foutenanalyse}
\subsection{Rechtstreekse grootheden}
Aangezien we de rechtstreeks gemeten grootheid $I$ éénmaal meten, kan er niets gezegd worden over de toevallige fout op $I$.
\subsection{Systematische fouten}
Bij opgave 3 werd de stroom rechtstreeks opgemeten met behulp van een digitale multimeter.
De opgegeven reading en digits bedraagt 1\% rdg + 1 dig.
De instrumentale fout wordt bepaald door 1\% van de afgelezen waarde te nemen en op te tellen met 0,1.
\\
Daarnaast zijn er ook systematische fouten op de lengte, breedte en dikte van de balk:
\begin{itemize}
    \item $L = (250 \pm 1)mm$
    \item $B = (17.32 \pm 0,11)mm$
    \item $D = (2,44 \pm 0,13)mm$
\end{itemize}
