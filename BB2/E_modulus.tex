\section{Bepaling E modulus}

We meten op analoge wijze het rekstrookje.
Dit is bij $R_a$ en $R_b$ gelijk aan $10\Omega$ en 
men bekomt dan de waarden in tabel \ref{tab:rekstrookje}.

\begin{table}[h]
    \centering
    \caption{Meetresultaten rekstrookje}
    \label{tab:rekstrookje}

    \begin{tabular}{| c | c | c | c | c |}
        \hline
        Kracht [N] & $R_{min} [\Omega]$& $R_{max} [\Omega]$& $I_{min}$ [mA] & $I_{max}$ [mA] \\ \hline
        0,981      & 130               & 120               & -0,041    & 0,003 \\ \hline
        1,962      & 130               & 120               &-0,047     & 0,003 \\ \hline
        2,943      & 130               & 120               &-0,04      & 0,004 \\ \hline
        3,924      & 130               & 120               &-0,04      & 0,004 \\ \hline
        4,905      & 130               & 120               &-0,039     & 0,004 \\ \hline
        5,886      & 130               & 120               &-0,039     & 0,004 \\ \hline
        6,867      & 130               & 120               &-0,038     & 0,005 \\ \hline
    \end{tabular}
\end{table}

Na interpolatie volgens volgende formule:

\begin{equation}
    R = R_{min} - \frac{\Delta R}{\Delta I} \cdot I_{min}
\end{equation}


Bekomt men R. Met Deze waarde wordt het mogelijk om $R_{x}$ te
berekenen via een schaalfactor 1 bepaald door $R_{a}$ en $R_{b}$.

$\Delta R_{x}$ wordt dan weer bepaald door het verschil met de
nominale waarde van het rekstrookje (waarde van de weerstand
bij evenwicht, ongeveer $120\Omega$).
\\ \\
Men bekomt de waarden uit tabel \ref{tab:interpol_rekstrookje}

\begin{table}[h]
    \centering
    \caption{Interpolatie rekstrookje}
    \label{tab:interpol_rekstrookje}
    \begin{tabular}{| c | c | c | c |}
        \hline
        R [$\Omega$]    & Rx [$\Omega$] & $\Delta R_{x}$    & $\frac{\Delta R_{x}}{R_{x}}$ \\ \hline
        120,682         & 120,682       & 0,682             & 0,005650 \\ \hline
        120,600         & 120,600       & 0,600             & 0,004975 \\ \hline
        120,909         & 120,909       & 0,909             & 0,007519 \\ \hline
        120,909         & 120,909       & 0,909             & 0,007519 \\ \hline
        120,930         & 120,930       & 0,930             & 0,007692 \\ \hline
        120,930         & 120,930       & 0,930             & 0,007692 \\ \hline
        121,163         & 121,163       & 1,163             & 0,009597 \\ \hline
    \end{tabular}
\end{table}

Nu kan men een lineaire regressie uitvoeren met als x-waarden
de krachten uitgeoefend op het balkje en als y-waarden
$\frac{\Delta R_{x}}{R_{x}}$.

Met de lineaire regressie functie in excel kan men nu de
richtingsco\"effici\"ent bepalen. Deze zal van volgende vorm
zijn:

\begin{equation}
    m = \frac{3}{2} \cdot \frac{K_1 \cdot L}{E \cdot b \cdot d^2}
\end{equation}

Na enige omvorming krijgt men een formule voor $E$:

\begin{equation}
    E = \frac{3}{2} \cdot \frac{K_1 \cdot L}{z \cdot b \cdot d^2}
\end{equation}

Samen met volgende fysische eigenschappen van het balkje:

\begin{table}[h]
    \centering
    \caption{Fysische eigenschappen balkje}
    \label{tab:fys_eig_balkje}
    \begin{tabular}{| c | c | c |}
        \hline
            & Waarde& Fout \\ \hline
        K1  & 2,110 & 0,01 \\ \hline
        L   & 0,250 & 0,00 \\ \hline
        b   & 0,020 & 0,00 \\ \hline
        d   & 0,005 & 0,00 \\ \hline
    \end{tabular}
\end{table}

Kan men nu $E$ en de fout hierop berekenen:

\begin{equation}
    E = 9,964E+09 \pm 4,7E+07
\end{equation}
