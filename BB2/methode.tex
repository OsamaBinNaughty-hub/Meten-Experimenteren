\section{Methode}

Om de apparatuur in het labo niet te vernietigen, maakt men
eerst een theoretische analyse met fictieve waarden en een 
perfecte ijking. Eens men dit heeft kan men voor de apparatuur
gevaarlijke waarden dus vermijden.
\\ \\
Als tweede stap gaat men de gevoeligheid van de gemaakte brug
van Wheatstone testen door een gekende weerstand nauwkeurig uit
te meten. Men kiest hiervoor een onbekende weerstand van
ongeveer $120 \Omega$. Verder zijn dit de volgende parameters
(men zal deze later nog uitdiepen):

% TODO: Dit moet eens nagekeken worden
\begin{table}[h]
    \caption{Omgevingswaarden van de proef}
    \label{tab:omgevingswaarden}
    \centering
    \begin{tabular}{| c | c | c |}
        \hline
        Grootheid       & Waarde    & Fout \\ \hline
        U [V]           & 1,518     & 10mV \\ \hline
        Rx [$\Omega$]   & 120       &      \\ \hline
        Rg [$\Omega$]   & 100       & 1E-7 \\ \hline
        $\Delta l$ [A]  & 1E-07     &      \\ \hline
    \end{tabular}
\end{table}

Als derde en laatste stap zal men de elasticiteit berekenen van
een balkje aan de hand van een rekstrookje en de gemaakte brug
uit voorgaande stap. De reden waarom men voor deze kiest zal
verder worden besproken.
