\section{Bespreking meetresultaten}

\begin{table}[H]
    \centering
    \label{tab:vergelijking}
    \caption{Vergelijking berekende resulaten met de echte waarden}
    \begin{tabular}{| c | c | c |}
        \hline
                                & Berekende waarde      & Echte waarde          \\ \hline
        Viscociteit             & $9.89 \cdot 10^{-4} Pa \cdot s$  & $9.775 \cdot 10^{-4}Pa \cdot s$ \\ \hline
        Kritisch Reynoldgetal   & $1800$                & ca. $2000$            \\ \hline
    \end{tabular}
\end{table}

De afschatting van de viscociteitsco\"effici\"ent verschilt $1.15 \cdot 10^{-5}$ met de echte waarde \cite{viscocity}.
Voor een betere schatting zouden we nog meer datapunten kunnen meten.
\\

We zien dat ons Reynoldgetal in het kritisch punt overeenkomt met de echte waarde, die ca. 2000 bedraagt. 
Deze schatting zou nog verbeterd kunnen worden door meer datapunten te meten rond het overgangspunt van 
laminair naar turbulent regime. \\

Om dit te verwezelijken kan tijdens het opmeten van de datapunten gekeken worden naar de opstelling om een
interval van $\Delta p$ te bepalen waarin het laminair regime nadert tot het overgangspunt. De twee factoren waarmee 
men rekening moet houden zijn de vorm van de uitvloeiende stroom en de variatie van het waterniveau in elke verticale buis.
Hoe meer het waterniveau varieert, hoe turbulenter de stroming.
