\section{Bespreking meetresultaten}

\begin{table}[H]
    \centering
    \label{tab:vergelijking}
    \caption{Vergelijking berekende resulaten met de echte waarden}
    \begin{tabular}{| c | c | c |}
        \hline
                                & Berekende waarde      & echte waarde          \\ \hline
        Viscociteit             & $9.89 \cdot 10^{-4}$  & $9.775 \cdot 10^{-4}$ \\ \hline
        Kritisch Reynoldgetal   & $1800$                & ca. $2000$            \\ \hline
    \end{tabular}
\end{table}

De afschatting van de viscociteitsco\"effici\"ent verschilt $1.15 \cdot 10^{-5}$ met de echte waarde \cite{viscocity}.
Voor een betere schatting zouden we nog meer datapunten kunnen meten, voornamelijk rond het overgangspunt van laminair
naar het turbulent regime.
\\

We zien dat ons Reynoldgetal in het kritisch punt overeenkomt met de echte waarde, die ca. 2000 bedraagt.