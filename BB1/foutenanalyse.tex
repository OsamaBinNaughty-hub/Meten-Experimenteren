\section{Foutenanalyse}

\subsection{Rechtsteekse grootheden}

Te bespreken grootheden:
\begin{itemize}
    \item Verschil in hoogte $\Delta h$
    \item De massa $m$ van het opgevangen water
    \item Tijd $\Delta t$ nodig om beker te vullen
    \item Temperatuur $T$ van het water
\end{itemize}

\subsubsection{Systematische fouten}
Er bestaan 2 systematische fouten, namelijk afleesfouten en instrumentale fouten. Bij digitale meettoestellen wordt de afleesfout als $\pm 1$ eenheid genomen op het meest rechtse digit. Bij analoge toestellen daarentegen wordt de afleesfout als $\pm \frac{1}{2}$ van de kleinste schaalverdeling.


\subsubsection{Toevallige fouten}

\subsection{Onrechtstreekse grootheden}

Te bespreken grootheden:
\begin{itemize}
    \item Debiet $Q$
    \item Verschil in druk $\Delta p$
\end{itemize}

\subsubsection{Systematische fouten}
\subsubsection{Toevallige fouten}

