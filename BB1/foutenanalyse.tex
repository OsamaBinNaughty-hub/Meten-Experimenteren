\section{Foutenanalyse}

\subsection{Rechtstreekse grootheden}

Te bespreken grootheden:
\begin{itemize}
    \item Verschil in hoogte $\Delta h$
    \item De massa $m$ van het opgevangen water
    \item Tijd $\Delta t$ nodig om de maatbeker te vullen
    \item Temperatuur $T$ van het water
\end{itemize}

\subsubsection{Systematische fouten}
Er bestaan twee systematische fouten, namelijk afleesfouten en
instrumentale fouten. Bij digitale meettoestellen wordt de afleesfout
als $\pm 1$ eenheid genomen op het meest rechtse digit. 
De afleesfout bij analoge toestellen daarentegen wordt
als $\pm\frac{1}{2}$ van de kleinste schaalverdeling genomen. 
\\

De instrumentale fout van een digitaal meettoetsel wordt opgegeven
door de constructeur. Aangezien deze nergens wordt vermeld,
kunnen we de instrumentale fout niet bepalen.
\\

De systematische fouten in dit practicum zijn:

\begin{itemize}
    \item $\Delta t$ werd gemeten aan de hand van een digitale chronometer. De afleesfout hierop bedraagt $\pm 0,01 s$.
    \item $m$ wordt bepaald door middel van een digitale balans, de afleesfout is gelijk aan $\pm 1g$
    \item Aangezien $\Delta h$ verkregen wordt door een relatieve meting, is de systematische fout gelijk aan nul.
    \item $T$ wordt gemeten door middel van een analoge thermometer. De afleesfout is $\pm \frac{1}{2} \cdot 1^{\circ}C = \pm 0,5^{\circ}C$.
\end{itemize} 

\subsubsection{Toevallige fouten}

%\begin{table}[h]
    %\centering
    %\label{tab:mDt}
    %\caption{Gemiddeldes en varianties voor $m$ en $\Delta t$}
    %\begin{tabular}{| c | c | c | c | c |}
        %\hline
         %& $\mu \{m\}$ & $\sigma ^2 \{m\}$ & $\mu \{\Delta t\}$ & $\sigma ^2 \{\Delta t\}$ \\ \hline
        %1 & 0,140000 &  0,000001 &  83,12 & 0,47 \\ \hline
        %2 & 0,13833333 &    9,3E-06 &   35,10 & 0,40 \\ \hline
        %3 & 0,138000 &	0,000013 &	31,15 &	0,50 \\ \hline
        %4 & 0,138000 &	7,0E-06 &	24,48 &	0,20 \\ \hline
        %5 & 0,142000 &	0,000019 &	22,12 &	0,83 \\ \hline
        %6 & 0,1416666 &	2,3E-06 &	20,06 &	0,38 \\ \hline
        %7 & 0,138333 &	2,0E-05 &	17,14 &	0,42 \\ \hline
        %8 & 0,13933333 &	3,3E-07 &	16,1500 &	0,0013 \\ \hline
        %9 & 0,1383333 &	6,3E-06 &	15,79 &	0,21 \\ \hline
        %10 & 0,143000 &	7,0E-06 &	13,410 &	0,063 \\ \hline
        %11 & 0,1416666 &	2,3E-06 &	11,8000 &	0,0050 \\ \hline
        %12 & 0,1383333 &	5,3E-06 &	78,61 &	0,78 \\ \hline
        %13 & 0,138666 &	3,4E-05 &	53,6 &	3,0 \\ \hline
        %14 & 0,14266666 &	3,3E-07 &	43 &	12 \\ \hline
        %15 & 0,1416666 &	6,3E-06 &	16,677 &	0,065 \\ \hline
        %16 & 0,141000 &	7E-06       &	15,253 &	0,100 \\ \hline
    %\end{tabular}
%\end{table}

Zowel $\Delta T$ en $\Delta h$ werden telkens maar \'e\'en keer gemeten,
bijgevolg kan er niets gezegd worden over de toevallige fout op beide grootheden.
\\

Men bevindt zich in het geval van gelijkwaardige grootheden aangezien elke meting
met hetzelfde meettoestel werd uitgevoerd. De toevallige fouten op $m$ en $\Delta t$
werden bepaald op basis van volgende formules:
\begin{equation}
    m = \frac{1}{n} \sum\limits_{i=1}^n x_i
\end{equation}

\begin{equation}
    \sigma^{2} \hat{=} \frac{\sum\limits_{i=1}^n (x_i - m)^2}{n - 1}
\end{equation}


%De uitkomsten van al deze berekingen kan men vinden 
%in tabel \ref{tab:mDt}.

\subsection{Onrechtstreekse grootheden}

Te bespreken grootheden:
\begin{itemize}
    \item Debiet $Q$
    \item Het drukverschil $\Delta p$
\end{itemize}

We berekenen het debiet $Q$ aan de hand van de formule: 
\begin{equation}
\label{debiet Q adhv rho, m en t}
    Q = \frac{\Delta m}{\rho \cdot \Delta t}
\end{equation}

%Dit maakt het debiet een onrechtstreeks gemeten grootheid.
\\

De druk $\Delta p$ wordt onrechtstreeks bepaald aan de hand
van de massadichtheid $\rho$, de gravitatieconstante $g$ en
het hoogteverschil $\Delta h$. Omdat $g$ en $\rho$ constante
waardes zijn en $\Delta h$ een relatieve meting is, is de
toevallige fout op deze grootheid gelijk aan nul.


\subsubsection{Systematische fouten}
De systematische fout op $Q$ wordt berekend aan de hand van:
\begin{equation}
    \Delta Y = \sum\limits_{i=1}^n \left|\frac{\partial \Psi}{\partial x_i}\right|_o \cdot \Delta x_i 
\end{equation}
Als men dit toepast op \eqref{debiet Q adhv rho, m en t} verkrijgt men:

\begin{equation*}
    \Delta Q = \left|\frac{\partial Q}{\partial m}\right|\cdot \Delta m + \left| \frac{\partial Q}{\partial \Delta t}\right| \cdot \Delta t
\end{equation*}

\begin{equation*}
    \Delta Q = \left|\frac{1}{\rho \cdot \Delta t}\right|\Delta m + \left|\frac{\Delta m}{\rho \cdot \Delta t^{2}}\right|\cdot \Delta t
\end{equation*}


\subsubsection{Toevallige fouten}
De toevallige fout op $Q$ is gelijk aan:
\begin{equation}
    \sigma^{2} \hat = \sum\limits_{i=1}^n(\frac{\partial \Psi}{\partial x_i})^{2} \cdot \sigma^{2} \{x_i\}
\end{equation}

In ons geval (gebruik makende van formule \eqref{debiet Q adhv rho, m en t}):
\begin{equation*}
    \sigma^{2}\{Q\} \hat = (\frac{\partial Q}{\partial m})^{2} \cdot \sigma^{2}\{m\} + (\frac{\partial Q}{\partial \Delta t})^{2} \cdot \sigma^{2}\{\Delta t\}
\end{equation*}
\begin{equation*}
\sigma^{2}\{Q\} = (\frac{1}{\rho \cdot \Delta t})^{2} \cdot \sigma^{2}\{m\} + (\frac{\Delta m}{\rho \cdot \Delta t^{2}})^{2} \cdot \sigma^{2}\{\Delta t\}
\end{equation*}





