\section{Foutenanalyse}

\subsection{Rechtsteekse grootheden}

Te bespreken grootheden:
\begin{itemize}
    \item Verschil in hoogte $\Delta h$
    \item De massa $m$ van het opgevangen water
    \item Tijd $\Delta t$ nodig om beker te vullen
    \item Temperatuur $T$ van het water
\end{itemize}

\subsubsection{Systematische fouten}
Er bestaan 2 systematische fouten, namelijk afleesfouten en instrumentale fouten. Bij digitale meettoestellen wordt de afleesfout als $\pm 1$ eenheid genomen op het meest rechtse digit. 
Bij analoge toestellen daarentegen wordt de afleesfout als $\pm \frac{1}{2}$ van de kleinste schaalverdeling genomen. De instrumentale fout wordt berekend aan de hand van de door de producent gegeven rdg en dig. Aangezien we deze niet terugvinden hebben we de instrumentale niet kunnen bepalen.
\begin{itemize}
    \item $\Delta t$ werd gemeten aan de hand van een digitale chronometer. De afleesfout hierop bedraagt $\pm 0,01 s$.
    \item $m$ werd bepaald door middel van een digitale balans, de afleesfout is gelijk aan $\pm 1g$
    \item Aangezien $\Delta h$ verkregen werd door een relatieve meting uit te voeren, is de systematische fout gelijk aan 0. Hierdoor is de afleesfout te verwaarlozen.
    \item $T$ werd gemeten door middel van een analoge thermometer. De afleesfout is $\pm \frac{1}{2} \cdot 1C = \pm 0,5C$.
\end{itemize} 



\subsubsection{Toevallige fouten}
$\Delta T$ en $\Delta h$ werden telkens maar 1 maal gemeten, bijgevolg kan er niets gezegd worden over de toevallige fout op deze 2 grootheden.
\\
We bevinden ons in het geval gelijkwaardige grootheden aangezien deze telkens met hetzelfde meettoestel gemeten werden.
De toevallige fouten op $m$ en $\Delta t$ werden bepaald op basis van volgende formules:
\begin{equation}
    m = \frac{1}{n} \sum\limits_{i=1}^n x_i
\end{equation}

\begin{equation}
    \sigma^{2} \hat{=} \frac{\sum\limits_{i=1}^n (x_i - m)^2}{n - 1}
\end{equation} 

Voor $m$:
\\$m = $ %hier moet onze bepaalde waarde komen
\\$\Delta \sigma^{2} = $ %hier moet onze bepaalde waarde komen 

Voor $\Delta t$:
\\$m = $ %hier moet onze bepaalde waarde komen
\\$\Delta \sigma^{2} = $ %hier moet onze bepaalde waarde komen



\subsection{Onrechtstreekse grootheden}

We berekenen het debiet $Q$ aan de hand van formule \eqref{eq:hagen}. Daarom is dit een onrechtstreeks gemeten grootheid.
De druk $\Delta p$ wordt onrechtstreeks bepaald aan de hand de massadichtheid $\rho$, de gravitatieconstante $g$ en het hoogteverschil $\Delta h$. Omdat $g$ en $\rho$ constante waardes zijn en $\Delta h$ een relatieve meting is, is de toevallige fout op deze grootheid gelijk aan nul.
Te bespreken grootheden:
\begin{itemize}
    \item Debiet $Q$
    \item Het drukverschil $\Delta p$
\end{itemize}

\subsubsection{Systematische fouten}
De systematische fout op $Q$ wordt berekend aan de hand van:
\begin{equation}
    \Delta Y = \sum\limits_{i=1}^n \left|\frac{\partial \Psi}{\partial x_i}\right|_o \cdot \Delta x_i
\end{equation}

\subsubsection{Toevallige fouten}

