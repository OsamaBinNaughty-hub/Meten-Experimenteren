\section{Foutenanalyse}

\subsection{Rechtsteekse grootheden}

Te bespreken grootheden:
\begin{itemize}
    \item Verschil in hoogte $\Delta h$
    \item De massa $m$ van het opgevangen water
    \item Tijd $\Delta t$ nodig om beker te vullen
    \item Temperatuur $T$ van het water
\end{itemize}

\subsubsection{Systematische fouten}
Er bestaan 2 systematische fouten, namelijk afleesfouten en instrumentale fouten. Bij digitale meettoestellen wordt de afleesfout als $\pm 1$ eenheid genomen op het meest rechtse digit. 
Bij analoge toestellen daarentegen wordt de afleesfout als $\pm \frac{1}{2}$ van de kleinste schaalverdeling te nemen. De instrumentale fout wordt berekend aan de hand van de door de producent gegeven rdg en dig. Aangezien we deze niet terugvinden hebben we de instrumentale niet kunnen bepalen.
\begin{itemize}
    \item $\Delta t$ werd gemeten aan de hand van een digitale chronometer. De afleesfout hierop bedraagt $\pm 0,01 s$.
    \item $m$ werd bepaald door middel van een digitale balans, de afleesfout is gelijk aan $\pm 1g$
    \item Aangezien $\Delta h$ verkregen werd door een relatieve meting uit te voeren, geldt de afleesfout op de meetlat niet.
    \item $T$ werd gemeten door middel van een analoge thermometer. De afleesfout is $\pm \frac{1}{2} \cdot 1C = \pm 0,5C$.
\end{itemize} 




\subsubsection{Toevallige fouten}
$\Delta T$ en $\Delta h$ werden telkens maar 1 maal gemeten, bijgevolg kan er niets gezegd worden over de toevallige fout op deze 2 grootheden.
\\
De toevallige fouten op $m$ en $\Delta t$ werden bepaald op basis van volgende formules:
\begin{equation}
    m = \frac{1}{n} \sum\limits_{i=1}^n x_i
\end{equation}

\begin{equation}
    \sigma^{2} \hat{=} \frac{\sum\limits_{i=1}^n (x_i - m)^2}{n - 1}
\end{equation} 

\subsection{Onrechtstreekse grootheden}

Te bespreken grootheden:
\begin{itemize}
    \item Debiet $Q$
    \item Verschil in druk $\Delta p$
\end{itemize}

\subsubsection{Systematische fouten}
\subsubsection{Toevallige fouten}

