\section{Theorie}

\subsection{Afleiding schuifkracht voor laminaire stroming en 
definitie van de viscositeit}

Als een voorwerp zich beweegt in een flu\"idum, dan verstoort het de snelheidsverdeling hiervan.
Vloeistofmoleculen die in onmiddelijk contact zijn met het voorwerp, krijgen door 
adhesiekrachten dezelfde snelheid, terwijl ver ervandaan het milieu niet verstoord wordt.\\

\textbf{HIER KOMT AFBEELDING}
\\

\textit{Figuur 1: Snelheidsverdeling in een viskeus midden tussen evenwijdige platen bij lage snelheden}\\

Voor lage snelheden beweegt de vloeistof zich in lagen over elkaar. Dit heet \textbf{laminaire stroming}.
De laag $z = z_{0}$ heeft de snelheid $v_{0}$ de snelheid nul heeft. De snelheid van de 
vloeistoflagen daartussenin verandert lineair in functie van de hoogte z.
\\

In het algemeen neemt de snelheid echter niet lineair toe met de dikte an de vloeistoflaag. 
De evenredigheidsconstante is ook afhankelijk van de aard van het flu\"idum, en wordt de
\textbf{viscositeitsco\"effici\"ent} genoemd. We bekomen de formule:
\\

$$\frac{F_{W}}{S} = \eta \cdot \frac{dv}{dn}$$
\\

Met
\\

$S=$ de oppervlakte van de vloeistoflagen

$n=$ de normaal loodrecht op dit oppervlak

$dv=$ de snelheidsverandering over de lengte

$dn=$ de normaal op S
\\

De \textbf{viscositeitsco\"effici\"ent} $\eta$ is dus de kracht per oppervlakte-eenheid nodig 
om snelheidsverschil van \'e\'en eenheid te handhaven tussen twee lagen vloeistof gelegen 
op een eenheidsfactor van elkaar.

\subsection{Laminaire stroming in een cilindervormige buis}

De wet van Hagen-Poiseuille geeft de snelheidsverdeling van de vloeistof 
in een horizontale cilindervormige buis bij stationaire, laminaire storming.
\\

$$v(r) = \frac{\pi(p_{1}-p_{2}) \cdot R^4}{4 \cdot L \cdot \eta}$$
\\
Het volumedebiet Q is het volume vloeistof dat per seconde door de buis stroomt.
Dit is te berekenen door die vergelijking over de doorsnede van de buis te integreren. 
Dit geeft: 

