\section{Methode}

Door gebruik te maken van de Hagen-Poiseuille opstelling (Figuur \ref{fig:hagen-pois}) 
kunnen we experimenteel de viscociteitsco\"effici\"ent van leidingwater bepalen. \\

\begin{figure}[h]
    \centering
    \caption{Hagen-Poiseuille opstelling}
    \includegraphics[width=0.7\textwidth]{img/hagen_poiseuille.png}
    \label{fig:hagen-pois}
\end{figure}

\subsection{Opstelling}

De opstelling bestaat uit twee onderdelen: een basin
met een waterpomp en een plastieken buizensysteem met
een invoer- en uitvoerkraan.
\\ \\
Het buizensysteem is enerzijds opgebouwd uit twee verticale 
buizen. Deze buizen zijn elk op een verschillende plaats 
verbonden met een U-vormige horizontale buis lager in de opstelling.
Het leidingwater in de onderste horizontale buis ondervindt een 
druk veroorzaakt door het leidingwater aanwezig in de twee verticale
kolommen en die luchtdruk die hierop speelt. Beide uiteinden van de 
U-vormige buis monden uit in het basin.

\subsection{Uitvoering}

Vooraleer metingen uitgevoerd kunnen worden, moet men ervoor zorgen 
dat er geen luchtbellen aanwezig zijn in het systeem. 
Dit kan bereikt worden door het stroomuiteinde en de pomp
ver genoeg van elkaar te plaatsen. Hiernaast moet de uitvoerkraan 
volledig dichtgedraaid worden.
\\ \\
Vervolgens gebruikt men de invoer- en uitvoerkranen om
het waterpeil in de verticale buizen te manipuleren tot
een zelfbepaalde hoogte. Het verschil in hoogte noemen we
$\Delta h$.
\\ \\
Per $\Delta h$ meet men het volumedebiet $Q$ van de
uitvoerkraan. Men houdt de maatbeker onder het uitstroomuiteinde 
tot het water een bepaald volume bereikt wordt.
Vervolgens weegt men de maatbeker. Het vulproces wordt gechronometreerd
en men zal worden aangeduid als $\Delta t$.
\\ \\
Men herhaalt deze procedure drie keer voor elk meetpunt. 
Er worden minstens zes meetpunten gemeten voor zowel laminaire
als turbulente stroming met nadruk op de overgangszone. In dit practicum
hebben we 16 punten gemeten.

\subsection{Fysische eigenschappen van opstelling}


\begin{table}[h]
    \centering
    \caption{Fysische eigenschappen van de opstelling}
    \begin{tabular}{| c | c | c |}
        \hline
                        & waarden   & fout       \\ \hline
        Lengte [m]      & 1,0060    & $\pm$ 0,0002 \\ \hline
        Diameter [m]    & 0,0040    & $\pm$ 0,0001 \\ \hline
        Temperatuur [C] & 21,0      & $\pm$ 0,5    \\ \hline
        
    \end{tabular}
\end{table}


