\section{Methode}

Door gebruik te maken van de Hagen-Poiseuille (Figuur \ref{fig:hagen-pois}) 
opstelling bepalen we experimenteel de viscociteitsconstante. \\

\begin{figure}[h]
    \centering
    \caption{Hagen-Poiseuille opstelling}
    \includegraphics[width=0.7\textwidth]{img/hagen_poiseuille.png}
    \label{fig:hagen-pois}
\end{figure}

\subsection{Opstelling}

De opstelling bestaat uit twee grote onderdelen: een basin
met een waterpompje en een glazen buizensysteem met
een invoer- en uitvoerkraantje.
\\ \\
De glazen opstelling bestaat dan weer uit 2 verticale 
glazen buizen. Verder heb je een glazen buis
die de verticale glazen buizen onderaan verbindt.
Het water in de onderste buis ondervindt hierdoor een druk
geïnduceerd door het water in de 2 verticale kolven en de
luchtdruk die op deze speelt.

\subsection{Uitvoering}

Eerst zorgt men er voor dat er geen luchtbellen aanwezig zijn, door
de pomp niet te dicht bij het uitstroomuiteinde te plaatsen
en het uitvoerkraantje helemaal dicht te draaien.
Dit om verkeerde metingen uit te sluiten.
\\ \\
Vervolgens gebruikt men de invoer- en uitvoerkraantjes om
het waterpeil in de verticale buizen te manipuleren tot
een zelfbepaalde hoogte. Het verschil in hoogte zullen we
$\Delta h$ noemen.
\\ \\
Per $\Delta h$ meet men het debiet $Q$ van het
uitvoerkraantje. Men houdt de maatbeker onder het uitstroomuiteinde 
tot een bepaald volume bereikt wordt.
Vervolgens weegt men de maatbeker. Het vulproces wordt gechronometreerd
en men zal hiernaar refereren via $\Delta t$.
\\ \\
Men herhaalt deze procedure voor een aantal gekozen meetpunten.
In dit practicum heeft men er 16 gemeten.

\subsection{Fysische eigenschappen van opstelling}


\begin{table}[h]
    \centering
    \caption{Fysische eigenschappen van de opstelling}
    \begin{tabular}{| c | c | c |}
        \hline
                        & waarden   & fout       \\ \hline
        Lengte [m]      & 1,0060    & $\pm$ 0,0002 \\ \hline
        Diameter [m]    & 0,0040    & $\pm$ 0,0001 \\ \hline
        Temperatuur [C] & 21,0      & $\pm$ 0,5    \\ \hline
        
    \end{tabular}
\end{table}


