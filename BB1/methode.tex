\section{Methode}

Door middel van een Hagen-Poiseuille opstelling
(Figuur \ref{fig:hagen-pois}), kunnen we enkele
versimpelingen bekomen. Zo zijn de glazen buizen
overal even rond (dezelfde diameter) en is de druk
op het water gemakkelijker te regelen.

\begin{figure}[h]
    \centering
    \caption{Hagen-Poiseuille opstelling}
    \includegraphics[width=0.7\textwidth]{img/hagen_poiseuille.png}
    \label{fig:hagen-pois}
\end{figure}

\subsection{Opstelling}

De opstelling bestaat uit uit 2 grote onderdelen: een basin
met een waterpompje en een glazen buizensysteem met
een invoer- en uitvoerkraantje.
\\ \\
De glazen opstelling bestaat dan weer uit 2 verticaal
opstaande glazen kolven. Verder heb je een glazen buis
die de opstaande glazen kolven van onder verbindt.
Het water in de onderste kolf ondervindt hierdoor een druk
geïnduceerd door het water in de 2 verticale kolven en de
luchtdruk die op deze speelt.
\\ \\
Men zorgt er voor dat alle lucht uit de kolven zijn door
eerst de opstelling vol te pompen met het waterpompje.
Dit om zekere fouten te vermijden (luchtdruk en dichtheid
zijn belangrijke componenten in de gebruikte formules).
\\ \\
Vervolgens gebruikt men de invoer- en uitvoerkraantjes om
het waterpeil in de verticale kolven te manipuleren tot
een zelfbepaalde hoogte. Het verschil in hoogte zullen we
$\Delta h$ noemen.
\\ \\
Per verschillende $\Delta h$ meet men het debiet $Q$ van het
uitvoerkraantje. Men doet dit door een beker onder het
kraantje te houden tot een gegeven volume en de massa $m$ van
het water te wegen. Het vulproces wordt gechronometreerd
en men zal hiernaar refereren via $\Delta t$.

\subsection{Fysische eigenschappen van opstelling}

\begin{table}
    \begin{tabular}{| c | c | c |}
        \hline
        
        Lengte [m]      & 1,0060    & 0,0002 \\ \hline
        Diameter [m]    & 0,0040    & 0,0001 \\ \hline
        Temperatuur [C] & 21,0      & 0,5    \\ \hline
        
    \end{tabular}
\end{table}
