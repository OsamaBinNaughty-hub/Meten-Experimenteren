\section{Berekening en fout viscociteitsconstante $\eta$}

De viscociteitsconstante in het laminaire regime is te berekenen met de formule van Hagen-Poiseuille:
\begin{equation}
    Q = \frac{\pi \cdot \Delta P \cdot R^2}{8 \cdot L \cdot \eta}
\end{equation}
Deze formule is van de vorm $$y=a\cdot x$$ met de rico $a$ gelijk aan: $$a = \frac{\pi \cdot R^2}{8 \cdot L \cdot \eta}$$
Uit deze formule is $\eta$ te berekenen.\\

Deze $\eta$ zullen we bepalen door een lineaire regressie toe te passen op de gemeten waarden van de laminaire regime.
We zitten ook in het geval van gelijkwaardige y-waarden.\\

Voor de lineaire regressie heb je het principe der kleinste kwadraten dat vereist
dat de $\chi ^2$-vorm minimaal is: 
\begin{equation}
    \chi ^2 = \sum\limits_{i=1}^n(y_i-a\cdot x_i)^2
\end{equation}
De beste schatting voor rico $a$ wordt dus gegeven door:
\begin{equation}
    a \hat{=} \frac{\sum\limits_{i=0}^n x_i  y_i}{\sum\limits_{i=0}^n x_i^2}
\end{equation}





