\section{Berekening Reynoldgetal}

Nu we de viscociteitsco\"effici\"ent hebben bepaald, kan men
hiermee de genormaliseerde, gemiddelde stroomsnelheid berekenen.
Dit staat ook wel beter bekend als het Reynoldgetal en wordt
berekend met de volgende formule:

\begin{equation}
    \label{eq:reynold}
    N_R = \frac{v\cdot \rho \cdot D}{\eta}
\end{equation}

Voor dit practicum is men enkel geïntereseerd in het Reynoldgetal
van het kritisch debiet. Dit punt is de maximale, stabiele, laminaire stroom,
net voor de overgang naar de turbulente stroom.\\

Vooraleer men deze waarde kan berekenen, moet men eerst het
kritische debiet $Q_{krit}$ bepalen. Dit om $v$ in de formule
\eqref{eq:reynold} in te vullen. Men doet dit door het snijpunt van 
de twee rechten te bepalen (zie figuur \ref{fig:grafiek}).

\begin{equation*}
    Q_{krit} = 5.58 \cdot 10^{-6}
\end{equation*}
$v$ kan nu berekend worden met volgende formule:
\begin{equation}
    v = \frac{Q_{krit}}{\pi \cdot R^2}
\end{equation}
\begin{equation*}
    v = 4.44 \cdot 10^{-1}
\end{equation*}
Tenslotte is het Reynoldgetal in het kritisch punt uit formule \eqref{eq:reynold} gelijk aan :
\begin{equation*}
    N_R = 1.80 \cdot 10^{3}
\end{equation*}
