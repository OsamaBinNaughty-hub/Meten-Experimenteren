\section{Berekening Reynoldgetal}

Nu we de viscociteitsconstante hebben bepaald, kunnen we het genormaliseerde gemiddelde stroomsnelheid berekenen
,genaamd het Reynoldgetal met de volgende formule:
\begin{equation}
    \label{eq:reynold}
    N_R = \frac{v\cdot \rho \cdot D}{\eta}
\end{equation}
Wij zijn geïntereseerd in het Reynoldgetal in het kritisch punt. Dit is het maximum stabiele
laminaire stroom, net voor de overgang naar de turbulente stroom.\\

Eerst moeten we het kritisch debiet bepalen om $v$ in formule \eqref{eq:reynold} te vullen.
Dit doen we door het snijpunt door onze 2 rechten te bepalen.
\begin{equation*}
    Q_{Qrit} = 5.58 \cdot 10^{-6}
\end{equation*}
Waardoor $v$ gelijk is aan:
\begin{equation}
    v = \frac{Q_{Qrit}}{\pi \cdot R^2}
\end{equation}
\begin{equation*}
    v = 4.44 \cdot 10^{-1}
\end{equation*}
Tenslotte is het Reynoldgetal in het kritisch punt uit formule \eqref{eq:reynold} gelijk aan :
\begin{equation*}
    N_R = 1.80 \cdot 10^{3}
\end{equation*}
