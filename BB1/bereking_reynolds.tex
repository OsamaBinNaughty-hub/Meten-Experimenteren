\section{Berekening Reynoldgetal}

Nu we de viscociteitsco\"effici\"ent hebben bepaald, kunnen we ermee de genormaliseerde
gemiddelde stroomsnelheid berekenen. Dit staat ook wel beter bekend als het 
\textbf{Reynoldgetal} en wordt berekend met de volgende formule:

\begin{equation}
    \label{eq:reynold}
    N_R = \frac{v\cdot \rho \cdot D}{\eta}
\end{equation}
Wij zijn enkel geïntereseerd in het Reynoldgetal van het kritisch debiet. Dit punt is  
het maximum stabiele laminaire stroom, net voor de overgang naar de turbulente stroom.\\

Vooraleer we deze waarde kunnen berekenen, moeten we eerst het kritische debiet $Q_{krit}$ bepalen
om $v$ in de formule \eqref{eq:reynold} in te vullen. Dit doen we door het snijpunt van 
onze twee rechten te bepalen.

\begin{equation*}
    Q_{krit} = 5.58 \cdot 10^{-6}
\end{equation*}
Waardoor $v$ gelijk is aan:
\begin{equation}
    v = \frac{Q_{krit}}{\pi \cdot R^2}
\end{equation}
\begin{equation*}
    v = 4.44 \cdot 10^{-1}
\end{equation*}
Tenslotte is het Reynoldgetal in het kritisch punt uit formule \eqref{eq:reynold} gelijk aan :
\begin{equation*}
    N_R = 1.80 \cdot 10^{3}
\end{equation*}
