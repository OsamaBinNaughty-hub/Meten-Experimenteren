\section{Foutenanalyse}
\subsection{Systematische fouten}
\subsubsection{Rechtstreekse grootheden}
De ampere meter is een analoog toestel waardoor de afleesfout gelijk is aan de helft van de kleinste schaalverdeling:
$$\pm\frac{1}{2}\cdot 2=1$$ De instrumentale fout staat vermeld op de amperemeter en is gelijk aan $0.5\% \cdot meetmereik$:
$$0,5\% \cdot 60 \cdot 2 = 0,6$$ Dit wil zeggen dat de absolute accuratie gelijk is aan$\pm 1,6$\\


\noindent Bij de digitate vermogenmeter is de afleesfout bij 10,20,30 Volt 
$\pm 1$ en bij 70V is deze gelijk aan $\pm 10$.
De instrumentale fout is gegeven door de fabrikant en gelijk aan $3\% + 2$\\
De absolute accuratie is hier berekend in de excel.\\

\noindent Op de digitale voltmeter is een afleesfout $\pm 0,1$ en de instrumentale fout gegegeven door de 
fabrikant gelijk aan $0,09\% + 2$\\
De absolute accuratie is hier ook automatisch berekend in de excel.\\

\noindent Op de RPM-meter is een afleesfout van $\pm 1$ en een instrumentale fout door de fabrikant van 
$0,05\% \cdot10000$. Dit is een percentage van het meetbereik.\\

\noindent Idem bij de koppelmeter is de afleesfout $\pm 1$ en de instrumentale fout van $0,05\% \cdot 500$.
Het meetbereik is hier normaalgezien van -500 tot 500, maar omdatdat we niet negatief mogen gaan in onze 
koppelmeting, nemen we een een meetbereik van 500.\\

\subsubsection{Onrechtstreekse grootheden}
De onrechtstreekse fout is gelijk aan:
\begin{equation}
    \Delta P = \left| \frac{\partial P}{\partial I}\right| \Delta I + \left| \frac{\partial P}{\partial U}\right| \Delta U
\end{equation}
\begin{equation*}
    \Delta P = U \cdot \Delta  I + I \cdot \Delta U
\end{equation*}
    





\subsection{Toevallige fouten}
\subsubsection{Rechtstreekse grootheden}
De $U_1,I_1,M,n$ en $P_{el2}$ werden rechtstreeks gemeten. De toevallige fouten
op deze grootheden werden berekend aan de hand van de volgende formule: 
\begin{equation}
    \sigma=\sqrt{\frac{\sum\limits_{i=0}^n(x_i-m)^2}{n-1}}  
\end{equation}

\subsubsection{Onrechtstreekse grootheden}
De onrechtstreekse grootheden in dit practica zijn $P_{el1},P_{mech},\eta_{1}$ en
$\eta_2$.\\
Volgende formules werden gebruikt om de toevallige fouten te bepalen:\\
\begin{equation}
    \sigma^2\{R\}=\sum\limits_{i=0}^n\left(\frac{\partial \Psi}{\partial X_i}\right)_0^2 \sigma ^2 \{X_i\}
\end{equation}
\begin{equation*}
    \sigma^2\{R\}=U^2 \cdot \sigma ^2 \{I\} + I^2 \cdot \sigma ^2 \{U\}
\end{equation*}

