\section{Gemeten waarden}

Men meet 4 punten: op 10A, 20A, 30A en 70A (vollast)
Op elk van deze punten wordt de ingangsstroom, ingangsspanning, mechanisch koppel,
toerental en uitgangsvermogen gemeten.
Vanuit deze metingen worden vervolgens het ingangsvermogen, het mechanisch vermogen
en het rendement op de 2 motoren/generatoren gemeten. Dit wordt gedaan door middel van
voorgaande formules.

\begin{tabular}{| c | c | c | c | c | c | c | c | c | c |}
    \hline
      & U$_1$  & I$_1$ & P$_{el1}$ & M     & n      & P$_{mech}$& P$_{el2}$ &$\eta_1$&$\eta_2$\\ \hline
    1 & 247,00 & 10,00 & 2470,00 & 15,00 & 157,08 & 2356,19 & 1350,00 & 0,9539 & 0,5730 \\ \hline
    2 & 241,20 & 10,00 & 2412,00 & 12,00 & 157,08 & 1884,96 & 1202,00 & 0,7815 & 0,6377 \\ \hline
    3 & 239,40 & 10,00 & 2394,00 & 12,00 & 157,08 & 1884,96 & 1164,00 & 0,7874 & 0,6175 \\ \hline
    4 & 238,30 & 10,00 & 2383,00 & 11,00 & 157,08 & 1727,88 & 1119,00 & 0,7251 & 0,6476 \\ \hline
    5 & 238,50 & 10,00 & 2385,00 & 13,00 & 157,08 & 2042,04 & 2213,00 & 0,8562 & 1,0837 \\ \hline
\end{tabular}

